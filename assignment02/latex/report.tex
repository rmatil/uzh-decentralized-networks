\documentclass{article}
\usepackage[utf8]{inputenc}
\usepackage[T1]{fontenc}
\usepackage{fullpage}
% \usepackage[left=3.0cm, right=2.5cm,top=2.0cm,bottom=2.5cm]{geometry}
\usepackage{enumerate}
%\usepackage{enumitem}
\usepackage{tabularx}
\usepackage{amsmath}

% \definecolor{dkgreen}{rgb}{0,0.6,0}
\usepackage{listings}
\usepackage{courier}
\usepackage{color}
\definecolor{mygreen}{rgb}{0,0.6,0}
\definecolor{mygray}{rgb}{0.5,0.5,0.5}
\definecolor{mymauve}{rgb}{0.58,0,0.82}

\usepackage[protrusion=true,expansion=true]{microtype} % Better typography

\lstset{
  language=bash,
  basicstyle=\small\ttfamily,
  captionpos=b,
  frame=none,
  literate=%
    {ö}{{\"o}}1
    {ä}{{\"a}}1
    {ü}{{\"u}}1
    {«}{{\guillemotleft}}1
    {»}{{\guillemotright}}1
}
\lstset{
  language=python,                % the language of the code
  backgroundcolor=\color{white},   % choose the background color
  basicstyle=\small\ttfamily,               % the size of the fonts that are used for the code
  breakatwhitespace=false,        % sets if automatic breaks should only happen at whitespace
  captionpos=b,                    % sets the caption-position to bottom
  breaklines=true,                % sets automatic line breaking
  showspaces=false,               % show spaces everywhere adding particular underscores; it overrides 'showstringspaces'
  showstringspaces=false,         % underline spaces within strings only
  showtabs=false,                 % show tabs within strings adding particular underscores
  commentstyle=\color{mygreen},   % comment style
  keywordstyle=\color{blue},      % keyword style
  stringstyle=\color{mymauve},     % string literal style
  numbers=left,
  stepnumber=1,
  frame=single,                   % adds a frame around the code
  frameround=,
  framerule=0.8pt,
}
\renewcommand{\labelenumi}{\alph{enumi}.}

\setlength\parindent{0pt} % Removes all indentation from paragraphs

%-------------------------------------------------------------------------------
% DOCUMENT INFORMATION
%-------------------------------------------------------------------------------

\title{Overlay Networks, Decentralized Systems and Their Applications
\\Exercise 2}

\author{Raphael \textsc{Matile}\\12-711-222\\raphael.matile@uzh.ch
\and Samuel \textsc{von Baussnern}\\09-914-623\\samuel.vonbaussern@uzh.ch}
\date{\today} % Date for the report

\begin{document}

\maketitle % Insert the title, author and date

\section{P2P Distributed Hash Tables}

\subsection{Tick the box with the right answer True (T), or Flase (F):}

\begin{description}
  \item[A central point will be always required with distributed indexing]
    False
  \item[A distributed hash table requires that each node has a unique identifier]
    True
  \item[P2P applications generate low amounts of traffic (compared to client-server)
        since that is the main purpose of having an P2P overlay]
    False
  \item[An overlay on top of another overlay is possible]
    True
\end{description}

\subsection{Cite and explain six key differences comparing the three strategies
discussed in the lecture slides to store and retrieve data in P2P systems
(Central server, Flooding search, Distributed indexing).}

TBD

\subsection{What is the complexity of each lookup strategy listed below?
And give a brief explanation of the reasons. Give the
complexity in big-O notation, e.g. O(N2).}

TBD

\subsection{How is addressing handled in DHTs? How are identifiers
chosen (for nodes and content)?}

TBD

\subsection{What happens if one node that is responsible for a given
DHT address space fails? Explain two possible techniques to overcome this problem.}

TBD

\subsection{Compare iterative routing with recursive routing.
Give at least one advantage and one disadvantage for each technique.}

TBD

\subsection{Cite method names, parameters, and return
values of the generic interface of distributed hash tables.}

TBD

\subsection{Considering Kademlia/TomP2P, answer the following:}
\subsubsection{How many IDs are possible?}

TBD

\subsubsection{Where is a key located?}

TBD

\subsubsection{What is the XOR distance between 3 and 4?}



\section{Challenge Task Preparation}

See file: TBD

\end{document}
