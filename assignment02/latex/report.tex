\documentclass{article}
\usepackage[utf8]{inputenc}
\usepackage[T1]{fontenc}
\usepackage{fullpage}
% \usepackage[left=3.0cm, right=2.5cm,top=2.0cm,bottom=2.5cm]{geometry}
\usepackage{enumerate}
%\usepackage{enumitem}
\usepackage{tabularx}
\usepackage{amsmath}

% \definecolor{dkgreen}{rgb}{0,0.6,0}
\usepackage{listings}
\usepackage{courier}
\usepackage{color}
\definecolor{mygreen}{rgb}{0,0.6,0}
\definecolor{mygray}{rgb}{0.5,0.5,0.5}
\definecolor{mymauve}{rgb}{0.58,0,0.82}

\usepackage[protrusion=true,expansion=true]{microtype} % Better typography

\lstset{
  language=bash,
  basicstyle=\small\ttfamily,
  captionpos=b,
  frame=none,
  literate=%
    {ö}{{\"o}}1
    {ä}{{\"a}}1
    {ü}{{\"u}}1
    {«}{{\guillemotleft}}1
    {»}{{\guillemotright}}1
}
\lstset{
  language=python,                % the language of the code
  backgroundcolor=\color{white},   % choose the background color
  basicstyle=\small\ttfamily,               % the size of the fonts that are used for the code
  breakatwhitespace=false,        % sets if automatic breaks should only happen at whitespace
  captionpos=b,                    % sets the caption-position to bottom
  breaklines=true,                % sets automatic line breaking
  showspaces=false,               % show spaces everywhere adding particular underscores; it overrides 'showstringspaces'
  showstringspaces=false,         % underline spaces within strings only
  showtabs=false,                 % show tabs within strings adding particular underscores
  commentstyle=\color{mygreen},   % comment style
  keywordstyle=\color{blue},      % keyword style
  stringstyle=\color{mymauve},     % string literal style
  numbers=left,
  stepnumber=1,
  frame=single,                   % adds a frame around the code
  frameround=,
  framerule=0.8pt,
}
\renewcommand{\labelenumi}{\alph{enumi}.}

\setlength\parindent{0pt} % Removes all indentation from paragraphs

%-------------------------------------------------------------------------------
% DOCUMENT INFORMATION
%-------------------------------------------------------------------------------

\title{Overlay Networks, Decentralized Systems and Their Applications
\\Exercise 2}

\author{Raphael \textsc{Matile}\\12-711-222\\raphael.matile@uzh.ch
\and Samuel \textsc{von Baussnern}\\09-914-623\\samuel.vonbaussern@uzh.ch}
\date{\today} % Date for the report

\begin{document}

\maketitle % Insert the title, author and date

\section{P2P Distributed Hash Tables}

\subsection{Tick the box with the right answer True (T), or Flase (F):}

\begin{description}
  \item[A central point will be always required with distributed indexing]
    False
  \item[A distributed hash table requires that each node has a unique identifier]
    True
  \item[P2P applications generate low amounts of traffic (compared to client-server)
        since that is the main purpose of having an P2P overlay]
    False
  \item[An overlay on top of another overlay is possible]
    True
\end{description}

\subsection{Cite and explain six key differences comparing the three strategies
discussed in the lecture slides to store and retrieve data in P2P systems
(Central server, Flooding search, Distributed indexing).}

\begin{description}
  \item[Single point of failure]

    Central server has it, as it stores the meta-data in a centralized node. The
    other strategies don't.

  \item[Network load]

    Central server and flooding search is not scalable, distributed indexing is.

  \item[Complexity of lookup strategy]

    Central server has a lookup complexity of $\mathcal{O}(1)$, flooding search
    $\mathcal{O}(n)$ and distributed indexing $\mathcal{O}(\log{}n)$.

  \item[All node state actualisation]

    Central server has a lookup complexity of $\mathcal{O}(n)$, flooding search
    $\mathcal{O}(1)$ and distributed indexing $\mathcal{O}(\log{}n)$.

  \item[Query fuzzy keys]

    Only possible with a centralized and flooding approach. Flooding might lead
    to different results though, and include false negatives.

  \item[Robustness]

    Distributed indexing and flooding search is robust, central server not.

\end{description}

\subsection{What is the complexity of each lookup strategy listed below? And
give a brief explanation of the reasons. Give the complexity in big-O notation,
e.g. $\mathcal{O}(N^2)$.}

\begin{description}
  \item[Central server]

    $\mathcal{O}(1)$

    As it stores the information directly. Additionally it requires only one
    request to fetch the meta-data.

  \item[Flooding]

    $\mathcal{O}(N)$

    As the request gets forwarded to every connected node but to the sender.

  \item[DHT]

    $\mathcal{O}(\log{}n)$

    The request is forwarded only to the nearest node, which is equivalent to a
    binary search.

\end{description}

\subsection{How is addressing handled in DHTs? How are identifiers chosen (for
nodes and content)?}

Each node gets a randomly determined key or the public key of the node for
addressing.

To identify certain resources the content is hashed.

\subsection{What happens if one node that is responsible for a given DHT address
space fails? Explain two possible techniques to overcome this problem.}

If the node fails to respond with a \lstinline{keep-alive} message a redundant
key-value pair stored on another node is chosen instead.

TBD

\subsection{Compare iterative routing with recursive routing. Give at least one
advantage and one disadvantage for each technique.}

In recursive routing the request is forwarded directly to the next node.
Adv.: You get a free update on the status of the traversed nodes.
Disadv.: It's not possible to track the process.

In iterative routing the request is transmitted by the searching node to each
node after the other until the request is complete.
Adv.: Tracking the progress.
Disadv.: You have to maintain a status list of each neighbour.

\subsection{Cite method names, parameters, and return values of the generic
interface of distributed hash tables.}

\begin{itemize}
  \item \lstinline{put :: (key, value) -> void}
  \item \lstinline{get :: key -> value}
\end{itemize}

\subsection{Considering Kademlia/TomP2P, answer the following:}
\subsubsection{How many IDs are possible?}

$2^{160}$

\subsubsection{Where is a key located?}

Keys are located on the node whose node ID is closet to the key.

\subsubsection{What is the XOR distance between 3 and 4?}

${011}_2\ \text{XOR}\ {100}_2 = {111}_2 = 7_{10}$

\section{Challenge Task Preparation}

See file: TBD

\end{document}
