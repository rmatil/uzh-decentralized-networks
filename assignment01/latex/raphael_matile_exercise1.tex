%%%%%%%%%%%%%%%%%%%%%%%%%%%%%%%%%%%%%%%%%
% University/School Laboratory Report
% LaTeX Template
% Version 3.0 (4/2/13)
%
% This template has been downloaded from:
% http://www.LaTeXTemplates.com
%
% Original author:
% Linux and Unix Users Group at Virginia Tech Wiki
% (https://vtluug.org/wiki/Example_LaTeX_chem_lab_report)
%
% License:
% CC BY-NC-SA 3.0 (http://creativecommons.org/licenses/by-nc-sa/3.0/)
%
%%%%%%%%%%%%%%%%%%%%%%%%%%%%%%%%%%%%%%%%%

%----------------------------------------------------------------------------------------
% PACKAGES AND DOCUMENT CONFIGURATIONS
%----------------------------------------------------------------------------------------

\documentclass{article}
\usepackage[utf8]{inputenc}
\usepackage[T1]{fontenc}
\usepackage{fullpage}
% \usepackage[left=3.0cm, right=2.5cm,top=2.0cm,bottom=2.5cm]{geometry}
\usepackage{enumerate}
%\usepackage{enumitem}
\usepackage{tabularx}
\usepackage{amsmath}

% \definecolor{dkgreen}{rgb}{0,0.6,0}
\usepackage{listings}
\usepackage{courier}
\usepackage{color}
\definecolor{mygreen}{rgb}{0,0.6,0}
\definecolor{mygray}{rgb}{0.5,0.5,0.5}
\definecolor{mymauve}{rgb}{0.58,0,0.82}

\usepackage[protrusion=true,expansion=true]{microtype} % Better typography

\lstset{
  language=bash,
  basicstyle=\small\ttfamily,
  captionpos=b,
  frame=none,
  literate=%
    {ö}{{\"o}}1
    {ä}{{\"a}}1
    {ü}{{\"u}}1
    {«}{{\guillemotleft}}1
    {»}{{\guillemotright}}1
}
\lstset{
  language=python,                % the language of the code
  backgroundcolor=\color{white},   % choose the background color
  basicstyle=\small\ttfamily,               % the size of the fonts that are used for the code
  breakatwhitespace=false,        % sets if automatic breaks should only happen at whitespace
  captionpos=b,                    % sets the caption-position to bottom
  breaklines=true,                % sets automatic line breaking
  showspaces=false,               % show spaces everywhere adding particular underscores; it overrides 'showstringspaces'
  showstringspaces=false,         % underline spaces within strings only
  showtabs=false,                 % show tabs within strings adding particular underscores
  commentstyle=\color{mygreen},   % comment style
  keywordstyle=\color{blue},      % keyword style
  stringstyle=\color{mymauve},     % string literal style
  numbers=left,
  stepnumber=1,
  frame=single,                   % adds a frame around the code
  frameround=,
  framerule=0.8pt,
}
\renewcommand{\labelenumi}{\alph{enumi}.}

\setlength\parindent{0pt} % Removes all indentation from paragraphs

%----------------------------------------------------------------------------------------
% DOCUMENT INFORMATION
%----------------------------------------------------------------------------------------

\title{Overlay Networks, Decentralized Systems and Their Applications\\Exercise 1} % Title

\author{Raphael \textsc{Matile}\\12-711-222\\raphael.matile@uzh.ch \and Samuel \textsc{von Baussnern}\\09-914-623\\samuel.vonbaussern@uzh.ch}
\date{\today} % Date for the report

\begin{document}

\maketitle % Insert the title, author and date

\section{Defnitions and Characteristics}
  \begin{enumerate}[1]
    \item \textbf{Define what is an «Overlay Network» in your own words, and give examples of applications that have/may have an overlay network.} \\
            An overlay network is a network built on top of another, 
            already existing network. It consists of multiple nodes which 
            are connected by links, either virtual or logical. 
            Each of these links represent a path on the underlying network. \\
            P2P Applications such as BitTorrent use the Internet as an underlying network to build their overlay network on top of it. VPN on top of IP.
    \item \textbf{What is the difference between «overlay» and «underlay»? Why is creating an overlay necessary?} \\
            «Overlay» means a layer which is located on top of another layer. 
            It lays \textit{over} the other. «Underlay», by contrast, means exactly the opposite. 
            The layer is located below another. \\
            Creating an overlay network might be necessary due to the use of a different addressing scheme for the individual nodes.
    \item \textbf{Discuss the following affirmation: «WhatsApp can be considered an Overlay Network.» Do you aggree/disagree? Explain the reasons.} \\
            In our opinion, WhatsApp can be considered as an Overlay Network because each user represents an individual node which 
            communicates with other nodes directly and the connection through the underlay is not visible. Also, WhatsApp provides a simple solution to address another node by using its phone number.
            Furhtermore, WhatsApp does use the Internet as an underlying network to build its services on top of it.
  \end{enumerate}

\section{Classification of P2P systems}
  \begin{enumerate}[1]
    \item \textbf{A full decentralized system should have «self-organization» and «direct interaction» properties. Why?}
          Self-organisation due to the possibility, that a particular node can suddenly be offline. If this node was a central service 
          and self-organisation would not be a requirement, then parts of the network or its entireness would fail to maintain its demanded service. \\
          It should use \textit{direct interaction} to limit its use of bandwidth, thus strengthening the stability and responsiveness of the system.
    \item \textbf{Skype can be considered a pure P2P network or an hybrid P2P network? Discuss the answer.}
          Skype is a hybrid P2P network, because each node has to verify its login credentials by a centralized authentication server. 
          Once a user resp. node is authenticated, Skype could be considered as a pure P2P network.
    \item \textbf{In file-sharing P2P applications, how is search/lookup done in “early” Napster (centralized)? And in Gnutella (flooding-based)? How about in BitTorrent (tracker- based)? Explain each with few short sentences, including at least one advantage and one disadvantage of each.}
          \begin{itemize}
            \item \textit{Napster}: There exists a centralized index service. Each node which requests a search for a file, contacts the 
                  index service, which provides a list of peers having the files available. 
                  \textit{Advantage}: Lookup of files can be guaranteed. 
                  \textit{Disadvantage}: The index service acts as a single point of failure.
            \item \textit{Gnutella}: A search request gets forwarded until a specific defined search depth is reached (ttl) or the search request gets answered.
                  If a node can answer the request, the search results are sent to the requesting peer. He can then download the file direclty from the offering entity.
                  \textit{Advantage}: Big effort is made to locate a particular file.
                  \textit{Disadvantage}: Not scalable, lookup can not be guaranteed.
            \item \textit{BitTorrent}: It is using a distributed hash table (DHT) for lookup of files. The network topology is formed such that for any key to a given hash, 
                  each node has a link to a node which owns the file to the key or a link to a node whose node ID is closer to the given key. Once there is no node with 
                  a ID closer to the key, the nearest node is reached which owns the file.
                  \textit{Advantage}: Highly scalable by automatically distributing loads to new nodes.
                  \textit{Disadvantage}: Very similar data values can be at totally different nodes due to the hash function.
          \end{itemize}
  \end{enumerate}

\section{P2P Applications}
  \begin{enumerate}[1]
    \item \textbf{Discuss why NAT is considered an enemy of P2P systems.} \\
          Because NAT allows hiding of nodes behind a single IP address. Because of that, a node can not be addressed directly anymore.
    \item \textbf{Imagine a P2P application that has an accurate locality system, i.e. that is able to identify the closest peers, considering “close” as the underlay hop distance (IP) between two nodes. Discuss why this property may be desirable by ISPs (Internet Service Providers).} \\
          ISPs can reduce the workload each node has to do overall as well as reduce the number of hops a packet needs to reach its target destination.
          Thus reducing the used bandwidth, power consumption and consumer satisfaction.

    \item \textbf{If you were affiliated with a copyright enforcement authority, how would you collect identities of users breaking copyright law, i.e. uploading illegal content in Gnutella? Would you conclude that users of P2P networks are generally less bound to be sued for distributing copyright-infringing material than users of a C/S system, such as RapidShare? What would be the best strategy for a user who does not want to get caught?}

          One way to identifiy peers in Gnutella is by sending search requests
          for files which are secured by copyright. Each peer which responds
          with search  results could then be identified. Since the content
          cannot be checked by a central authority corrupting the content with
          some sort of tracking device (information, program, etc.) is simple
          and effective.

          We would not conclude that P2P network users are generally less
          bound to be sued. Governments or attorneys might monitoring a P2P
          network  much heavier than a ordinary Client/Server service due to
          the knowledge that many users may use such networks to share
          copyrighted material. But the effort which has to be done to finally
          bring a user to justice might be higher than in a Client/Server
          infrastructre,  where the operator of such a service could simply be
          bound by law to prevent uploading copyrighted material.
          
          A possible strategy to not get caught could include the use of VPNs
          through multiple servers located in different countries. Because of
          that, one must work together with multiple different laws of
          different countries to obtain some information about an user.


  \end{enumerate}




\end{document}
