\documentclass{article}
\usepackage[utf8]{inputenc}
\usepackage[T1]{fontenc}
\usepackage{fullpage}
% \usepackage[left=3.0cm, right=2.5cm,top=2.0cm,bottom=2.5cm]{geometry}
\usepackage{enumerate}
%\usepackage{enumitem}
\usepackage{tabularx}
\usepackage{amsmath}

% \definecolor{dkgreen}{rgb}{0,0.6,0}
\usepackage{listings}
\usepackage{courier}
\usepackage{color}
\definecolor{mygreen}{rgb}{0,0.6,0}
\definecolor{mygray}{rgb}{0.5,0.5,0.5}
\definecolor{mymauve}{rgb}{0.58,0,0.82}

\usepackage[protrusion=true,expansion=true]{microtype} % Better typography

\lstset{
  language=Java,                % the language of the code
  backgroundcolor=\color{white},   % choose the background color
  basicstyle=\small\ttfamily,               % the size of the fonts that are used for the code
  breakatwhitespace=false,        % sets if automatic breaks should only happen at whitespace
  captionpos=b,                    % sets the caption-position to bottom
  breaklines=true,                % sets automatic line breaking
  showspaces=false,               % show spaces everywhere adding particular underscores; it overrides 'showstringspaces'
  showstringspaces=false,         % underline spaces within strings only
  showtabs=false,                 % show tabs within strings adding particular underscores
  commentstyle=\color{mygreen},   % comment style
  keywordstyle=\color{blue},      % keyword style
  stringstyle=\color{mymauve},     % string literal style
  numbers=left,
  stepnumber=1,
  frame=single,                   % adds a frame around the code
  frameround=,
  framerule=0.8pt,
  tabsize=4
}
\renewcommand{\labelenumi}{\alph{enumi}.}

\setlength\parindent{0pt} % Removes all indentation from paragraphs

%-------------------------------------------------------------------------------
% DOCUMENT INFORMATION
%-------------------------------------------------------------------------------

\title{Overlay Networks, Decentralized Systems and Their Applications
\\Exercise 3}

\author{Raphael \textsc{Matile}\\12-711-222\\raphael.matile@uzh.ch
\and Samuel \textsc{von Baussnern}\\09-914-623\\samuel.vonbaussern@uzh.ch}
\date{\today} % Date for the report

\begin{document}

\maketitle % Insert the title, author and date

\section{Bloom Filters}

\subsection{Which operations does the traditional Bloom Filter support?}

  \begin{itemize}
    \item Insertion
    \item Query
  \end{itemize}

\subsection{Does a Bloom Filter have a capacity limit? What changes if more and more elements are added?}

  \begin{itemize}
    \item A Bloom Filter does not have a capacity limit. So the add operation will never fail.
    \item The false positive rate increases as elements gets added.
  \end{itemize}

\subsection{What is a false positive? Explain how it can happen.}

  The false positive describes the possibility that a element can be in the given
  set of elements despite it should not be there at all.
  Since certain bits get set to 1 the likelihood that a certain query on the set
  matches exactly some bits of other hashed values increases.

\subsection{Can the traditional Bloom Filter have false negatives? Explain}

  No, it can not. Because bits do not get changed back to zero in insertion
  operation, all bits of inserted values remain at 1.

\subsection{Describe a real life application scenario for Bloom
Filters and explain why they are useful in the chosen scenario.}

  Google Chrome once used a Bloom Filter to identify malicious URLs. Of each
  website which should be fetched, the URL got checked against a local Bloom Filter.
  If a match was detected a complete lookup of the URL was made.

  This application of a Bloom Filter was useful, due to the avoidance of
  many complete lookups of URLs and the corresponding time cap.


\section{Kademlia}

\subsection{How many IDs are possible?}

  $2^{160}$ because 160 buckets are used

\subsection{Where is a key located?}

  Each node maintains a list of neighbours which maintain a range of keys.
  So keys are located on the node whose node ID is closest to the key.

\subsection{What is the XOR distance between 3 and 4?}

  ${011}_2\ \text{XOR}\ {100}_2 = {111}_2 = 7_{10}$

\subsection{Kademlia routing tables consist of a list for each bit of the node ID. E.g. if
a node ID consists of 128 bits, a node will keep 128 such lists. In this case, would 127
lists be enough, why?}

  Yes, it is enough, since the node has not maintain itself on the key list.

\section{Challenge Task Preparation}
See attached zip file.

\lstinputlisting[language=Java]{../src/exercise3/src/net/tomp2p/exercise/raphaelmatile/Main.java}
\lstinputlisting[language=Java]{../src/exercise3/src/net/tomp2p/exercise/raphaelmatile/DHT.java}

\end{document}
